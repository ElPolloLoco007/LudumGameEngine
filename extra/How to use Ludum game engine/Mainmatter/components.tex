\chapter{Components}
\section{Entity}

All objects that the user wants to create get created via the Entity class.



The constructor of Entity takes these arguments:

\begin{itemize}
	\item name
    \item body
    \item physics
    \item collisionDetection
    \item audioManager
    \item sprite
    \item imageRender
\end{itemize}


Example of how to create a $Entity$:

\begin{lstlisting}
class Box {
  constructor(x, y, height, width) {
    this.entity = new Entity(
      "Ludum",
      new Body(this, x, y, height, width),
      new Physics(this, 10, -6),
      new CollisionDetection(this),
      null,
      null,
      new ImageRender(this, ResourceManager.getImagePath("logo.png"))
    );
  }
}
\end{lstlisting}

Entity has these methods. These methods return either the component or null.

\begin{itemize}
	\item getEntity() returns entity
	\item getName() returns name of entity as a string
	\item getBody() returns body
	\item getPhysics() return physics
	\item getCollisionDetection() return CollisionDetection
	\item getAudioManager() returns audioManager
	\item getSprite() returns sprite
	\item getImgae() returns image	
	\item getUpdate() updates body of entity through physics
	\item getEntityProps() returns the newest value in body and physics
\end{itemize}


\section{Body}
Body class is the body of the entity.  

The constructor of Body takes these arguments:

\begin{itemize}
	\item entity
    \item left
    \item top
    \item height
    \item width
\end{itemize}

The body class contains only setters and getters for these parameters.

\begin{lstlisting}
class Object {
  constructor() {
    this.entity = new Entity(
      "Object",
      new Body(this, 300, 540, 100, 100),
      }
}
\end{lstlisting}

Here is a small example of how to move the entity bird:
\begin{lstlisting}

if (this.getBody().getTop() > 1040) {
      this.getBody().setTop(400);
      this.getBody().setLeft(300);
    }
\end{lstlisting}

\section{CollisionDetection}
To check for collitionDectection use:
\begin{lstlisting}
  checkForCollision(otherEntity)
\end{lstlisting}

Example of this can be:
\begin{lstlisting}

let hasPlayerCollided = player
          .getCollisionDetection()
          .checkForCollision(this.state.playerArr[index].getEntity());
\end{lstlisting}


\section{Physics}

Physics takes 3 arguments:

\begin{itemize}
	\item entity
    \item left
    \item top
\end{itemize}

Physics has setters for these items, while getters for left and top.

Physics also has a update arrow function. This arrow function is used to update the physics of the object from old to new position. 

This method is used in Entity update.

Example:

\begin{lstlisting}
 placeholder[0].getPhysics()
.setTop(this.state.playerArr[0].getPhysics().getTop() * -1);
\end{lstlisting}

This example is about changing the vertical direction of an object with physics.

\section{AudioManager}
example to use audioManager:

Object:
\begin{lstlisting}
    new AudioManager([
        ResMan.getAudioPath("soundEffect1.mp3"),
        ResMan.getAudioPath("soundEffect2.mp3"),
        ResMan.getAudioPath("soundEffect3.mp3")
      ]),    
    this.enum = {
      BIRD_JUMPS: 0,
      BIRD_SCORE: 1,
      BIRD_DIES: 2
};
\end{lstlisting}

Then if something happens:
\begin{lstlisting}

object.getAudioManager().play(2); // testing!!!
\end{lstlisting}


\section{Sprite}


\section{ImageRender}
To use ImageRender, simply add:

\begin{lstlisting}
new ImageRender(this, ResourceManager.getImagePath("logo.png"))
\end{lstlisting}

To an entity.

Example:
\begin{lstlisting}
class Box {
  constructor(x, y, height, width) {
    this.entity = new Entity(
      "Ludum",
      new Body(this, x, y, height, width),
      new Physics(this, 10, -6),
      new CollisionDetection(this),
      null,
      null,
      new ImageRender(this, ResourceManager.getImagePath("logo.png"))
    );
  }
\end{lstlisting}

To get this to work, you need to add:
\begin{lstlisting}

 getImage() {
    return this.entity.getImage();
  }
  // rendering this class
  render() {
    return <span className="frame">{this.getImage().render()}</span>;
  }
\end{lstlisting}

To the object file.
\section{ResourceManager}

To use the ResourceManager simply import the class and use it like this:

\begin{lstlisting}
ResourceManager.getImagePath("background.png")
\end{lstlisting}

\begin{lstlisting}
ResourceManager.getAudioPath("one.mp3")
\end{lstlisting}

\begin{lstlisting}
ResourceManager.getSpritePath("birds.png")
\end{lstlisting}

The $ResourceManager$ uses a default paths:

\begin{itemize}
	\item ../resources/image/
    \item ../resources/audio/
    \item ../resources/sprite/
\end{itemize}

You can change these paths in the $resourceManager$ file

\section{Background}

To use the $Background$ component, simply add it to the component.

\begin{lstlisting}
<Background
              height={1080}
              width={1920}
              speed={0.5}
              image={ResourceManager.getImagePath("background.png")}
            >
              {" "}
            </Background>{" "}
\end{lstlisting}

$ackground$ contains $defaultProps$, so it is not needed to set $height$, $width$ and $speed$. To set a image you can either use the $ResourceManager$ or simply import a image.

\section{HUD}
To use the HUD simply add the HUD component
\begin{lstlisting}
<HUD score={this.state.score} position={"tc"} />{" "}
\end{lstlisting}

Where score is a score variable from the game.

\section{Menu}
To use the menu, simply import the component into the file you want.

\begin{lstlisting}
<Menu showMenu={this.state.showMenu}
\end{lstlisting}

Using ${this.state.showMenu}$ gives you the option of toggling it on and off, depending of a boolean $showMenu$.

To add more menu options, simply add more items into the $menuItems$, and use $handleClick(e)$ for the event.

\section{ScoreBoard}
To use scoreboard simply add:

\begin{lstlisting}
<ScoreBoard />
\end{lstlisting}

To get the score from the game, $context$ is recommended, as it is shown in the $flappy$ demo, since normally a $menuItem$ shows a scoreboard.

\section{Logger}
To use the logger simply use:

\begin{lstlisting}
Logger.setText("flappy.js", `score: ${this.state.score}`);
\end{lstlisting}

where first argument is the name of file and second argument is what you want to log.

Then you can add this to the game:
\begin{lstlisting}
<LoggerManager /> 
\end{lstlisting}

  