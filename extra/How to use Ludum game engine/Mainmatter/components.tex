\chapter{Components}
\section{Entity}

All objects that the user wants to create get created via the Entity class.



The constructor of Entity takes these arguments:

\begin{itemize}
	\item name
    \item body
    \item physics
    \item collisionDetection
    \item audioManager
    \item sprite
\end{itemize}


Example of how to create a $Entity$:

\begin{lstlisting}
class Pipe {
  constructor(startPos, topPos, height, width) {
    this.len;
    this.entity = new Entity(
      "Bottom pipe",
      new Body(this, 1920 + startPos, topPos, height, width),
      new Physics(this, -8.85, 0),
      new CollisionDetection(this),
      null
    );
  }
}
\end{lstlisting}


\section{Body}
Body class is the body of the entity.  

The constructor of Body takes these arguments:

\begin{itemize}
	\item entity
    \item left
    \item top
    \item height
    \item width
\end{itemize}

The body class contains only setters and getters for these parameters.

\begin{lstlisting}
class Bird {
  constructor() {
    this.entity = new Entity(
      "Bird",
      new Body(this, 300, 540, 100, 100),
      }
}
\end{lstlisting}

Here is a small example of how to move the entity bird:
\begin{lstlisting}

if (this.getBody().getTop() > 1040) {
      this.getBody().setTop(400);
      this.getBody().setLeft(300);
    }
\end{lstlisting}

\section{CollisionDetection}
\section{Physics}
\section{AudioManager}
\section{Sprite}
\section{ResourceManager}
\section{Background}
\section{Menu}
\section{ScoreBoard}

